%%%% Originalvorlage von Kai Vogelgesang
\documentclass{article}
\usepackage{iftex}
\ifluatex
	\usepackage{fontspec}
\else
	\usepackage[utf8]{inputenc} %Eingabekodierung nach UTF8-Unicode, Erlaubt direkte Eingabe von Umlauten etc.
\fi
\usepackage{amssymb}
\usepackage{amsmath}
\usepackage{amsthm}
\usepackage[ngerman]{babel}
\usepackage{csquotes}

\usepackage[headheight=24pt,heightrounded]{geometry}
\usepackage{graphicx}
\usepackage{titlesec}
\usepackage{fancyhdr}
\usepackage{enumerate}

\usepackage{tikz}
\usetikzlibrary{automata}

\usepackage{xcolor}
\definecolor{urlcolor}{RGB}{0,136,204}
\definecolor{linkcolor}{RGB}{204,0,34}
\usepackage[colorlinks,urlcolor=urlcolor,linkcolor=linkcolor]{hyperref}

% TODOs
\usepackage[colorinlistoftodos,prependcaption,textsize=tiny]{todonotes}

% coloneqq
\usepackage{mathtools}

% Blank lines instead of indented first line
\setlength{\parindent}{0pt}
\setlength{\parskip}{\baselineskip}

\newcommand{\sheet}[1]{
\pagestyle{fancy}
\fancyhead[L]{Introduction to Theoretical Computer Science WS2024/25\\Exercise Sheet #1\\November 1st 2024}
\fancyhead[R]{Akansha Jain (7014299)\\Moustafa Said (7024414)\\Francesco Georg Schmitt (2574611)}

\titleformat{\section}
{\normalfont\Large\bfseries} {E#1.\thesection} {0.5em}{}
\titleformat{\subsection}
{\normalfont\large\bfseries} {}{1em}{}

}

% Hier kommen Vorlesungs- und Dokumentspezifische Makros

\newcommand{\NN}{\mathbb{N}}
\newcommand{\ZZ}{\mathbb{Z}}
\newcommand{\QQ}{\mathbb{Q}}
\newcommand{\RR}{\mathbb{R}}
\newcommand{\CC}{\mathbb{C}}
\newcommand{\inv}{^{-1}}
\newcommand{\lpar}{\left(}
\newcommand{\rpar}{\right)}

\newcommand{\limn}{{\lim_{n\to\infty}}}

\sheet{12}{January 24th, 2025}
\begin{document}

\section{\textsc{NP}-Reductions}

Show the reduction $\text{IS}\leq_{P}\text{Clique}$.

$$\text{IS}:=\left\{(G,k)\mid G\text{ is an undirected graph with a }k\text{-independent set}\right\}$$
$$\text{Clique}:=\left\{(G,k)\mid G\text{ is an undirected graph with a }k\text{-clique}\right\}$$

To prove the reduction $\text{IS}\leq_{P}\text{Clique}$, we need to find a function which maps inputs of \texttt{IS} to inputs of \texttt{Clique}.\\
Consider a graph $G = (V,E)$. Build the complement graph $\overline{G} = (V, \overline{E})$ where $\forall v_{1}, v_{2} \in V: (v_{1},v_{2}) \in \overline{E} \leftrightarrow (v_{1},v_{2}) \notin E$. This is obviously computable in polynomial time, by checking for every possible pairing of two nodes whether it is in the original edge set or not.\\
We do not need to modify the number $k$ in this case.

Written out formally, our reduction function is:
$$f(G,k) = (\overline{G},k)\text{ where }\overline{G}=(V,\overline{E})$$

To prove the correctness, we must show two separate statements:
\begin{itemize}
    \item $G$ has a $k$-independent set $\rightarrow$ $\overline{G}$ has a $k$-clique
    \item $G$ has no $k$-independent set $\rightarrow$ $\overline{G}$ has no $k$-clique
\end{itemize}
For the first case, we assume that $G$ has a $k$-independent set. This is, by definition, a subset $I\subseteq V$ of nodes $I = \left\{v_{1}, v_{2}, \ldots, v_{k}\right\}$ where $\forall v_{1},v_{2} \in I: (v_{1},v_{2})\notin E$. By the definition of the complement graph stated above, this directly implies $\forall v_{1},v_{2} \in I: (v_{1},v_{2})\in \overline{E}$. This directly implies that $\overline{E}$ has a $k$-clique (namely $I$), so we are done with this case.\\
We prove the second case by contraposition, so we attempt to show that\\
\phantom{x}\hspace{0.4cm}$G$ has a clique $\rightarrow$ $\overline{G}$ has an independent set\\
holds.\\
$G$ has a $k$-clique, which by definition means that there is a subset $C\subseteq V$ of nodes $C = \left\{v_{1}, v_{2}, \ldots, v_{k}\right\}$ where $\forall v_{1},v_{2} \in C: (v_{1},v_{2})\in E$. By the definition of the complement graph stated above, this directly implies $\forall v_{1},v_{2} \in V: (v_{1},v_{2})\notin \overline{E}$. This directly implies that $\overline{E}$ has a $k$-independent set (namely $C$), so we are done with this case.

We have provided a reduction function, shown that it runs in polynomial time and shown its correctness, so we have indeed shown $\text{IS}\leq_{P}\text{Clique}$.

\end{document}
