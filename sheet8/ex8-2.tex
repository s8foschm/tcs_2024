%%%% Originalvorlage von Kai Vogelgesang
\documentclass{article}
\usepackage{iftex}
\ifluatex
	\usepackage{fontspec}
\else
	\usepackage[utf8]{inputenc} %Eingabekodierung nach UTF8-Unicode, Erlaubt direkte Eingabe von Umlauten etc.
\fi
\usepackage{amssymb}
\usepackage{amsmath}
\usepackage{amsthm}
\usepackage[ngerman]{babel}
\usepackage{csquotes}

\usepackage[headheight=24pt,heightrounded]{geometry}
\usepackage{graphicx}
\usepackage{titlesec}
\usepackage{fancyhdr}
\usepackage{enumerate}

\usepackage{tikz}
\usetikzlibrary{automata}

\usepackage{xcolor}
\definecolor{urlcolor}{RGB}{0,136,204}
\definecolor{linkcolor}{RGB}{204,0,34}
\usepackage[colorlinks,urlcolor=urlcolor,linkcolor=linkcolor]{hyperref}

% TODOs
\usepackage[colorinlistoftodos,prependcaption,textsize=tiny]{todonotes}

% coloneqq
\usepackage{mathtools}

% Blank lines instead of indented first line
\setlength{\parindent}{0pt}
\setlength{\parskip}{\baselineskip}

\newcommand{\sheet}[1]{
\pagestyle{fancy}
\fancyhead[L]{Introduction to Theoretical Computer Science WS2024/25\\Exercise Sheet #1\\November 1st 2024}
\fancyhead[R]{Akansha Jain (7014299)\\Moustafa Said (7024414)\\Francesco Georg Schmitt (2574611)}

\titleformat{\section}
{\normalfont\Large\bfseries} {E#1.\thesection} {0.5em}{}
\titleformat{\subsection}
{\normalfont\large\bfseries} {}{1em}{}

}

\input{macros}
\sheet{8}{December 13th, 2024}
\setcounter{section}{1}
\begin{document}

\section{Reductions$++$}

\subsection{(a)}

We can define a function, where $g, z \in \mathbb{N}$:
\[
f(g, z) :=
\begin{cases}
    1 & \text{if } g = z, \\
    \text{undefined} & \text{otherwise}.
\end{cases}
\]

This is obviously \textbf{WHILE}-computable as a \textbf{WHILE}-program that enters an infinite loop if $g \neq z$. Since $f(g, z)$ is \textbf{WHILE}-computable, then by the recursion theorem, there is a $g \in \mathbb{N}$ such that $\varphi^1_g(g) = f(g, z)$ for all $z \in \mathbb{N}$.

Now, we have that $\varphi^1_g(g)$ is defined, and for everything else, $\varphi^1_g(g)$ is undefined for all $a \neq g$. Therefore, $\text{dom } \varphi_g = \{g\}$, and $L$ is not empty.


\subsection{(b)}

We show that $L \in \text{RE}$ by finding a reduction $\overline{H_0} \leq L$. We use the $s$-$m$-$n$ theorem for this.

Consider the function $f: \mathbb{N}^2 \to \mathbb{N}$ defined by:
\[
f(g, m) :=
\begin{cases}
    0 & \text{if } \text{göd}^{-1}(g) \text{ doesn't halt on } g \text{ after } m \leq \text{steps}, \\
    \text{undefined} & \text{otherwise}.
\end{cases}
\]

$f$ is obviously \textbf{WHILE}-computable. We can use the clocked version of the universal \textbf{WHILE}-program.

Let $e$ be a Gödel number of $f$. By the $s$-$m$-$n$ theorem:
\[
f(g, m) = \varphi^2_{(e, g)}(m) = \varphi_{S^1_1(e, g)}(m)
\]
for all $g, m \in \mathbb{N}$. But by construction, we have:

- If $g \in \overline{H_0}$:
  \[
  \text{göd}^{-1}(g) \text{ doesn't halt on input } g \iff f(g, m) = 0 \text{ for some } m \in \mathbb{N},
  \]
  \[
  \iff S_1(e, g) \in L.
  \]

- If $g \notin \overline{H_0}$:
  \[
  \text{göd}^{-1}(g) \text{ halts on input } g \iff f(g, m) \text{ is undefined for all } m \in \mathbb{N},
  \]
  \[
  \iff S_1(e, g) \notin L.
  \]

Thus, $S_1(e, \cdot)$ is the desired reduction, and $L \in \text{RE}$ because $\overline{H_0} \leq L$ and $\overline{H_0} \notin \text{RE}$.


\subsection{(c)}


In the same way, we find a reduction $H_0 \leq L$. From this, it follows that $L \notin \text{co-RE}$ because $H_0 \notin \text{co-RE}$.

\bigskip

Consider the following function $f: \mathbb{N}^2 \to \mathbb{N}$ defined by:
\[
f(g, m) :=
\begin{cases}
    0 & \text{if } \text{göd}^{-1}(g) \text{ halts on input } g \text{ after } m \leq \text{steps}, \\
    \text{undefined} & \text{otherwise}.
\end{cases}
\]

Let $e$ be a Gödel number of $f$. By the $s$-$m$-$n$ theorem:
\[
f(g, m) = \varphi^2(e, g)(m) = \varphi_{S_1(e, g)}(m)
\]
for all $g, m \in \mathbb{N}$. But by construction, we have:

- If $g \in H_0$:
  \[
  \text{göd}^{-1}(g) \text{ halts on input } g \iff f(g, m) = 0 \text{ for some } m \in \mathbb{N},
  \]
  \[
  \iff S_1(e, g) \in L.
  \]

- If $g \notin H_0$:
  \[
  \text{göd}^{-1}(g) \text{ doesn't halt on input } g \iff f(g, m) \text{ is undefined for some } m \in \mathbb{N},
  \]
  \[
  \iff S_1(e, g) \notin L.
  \]

Thus, $S_1(e, \cdot)$ is the desired reduction, and $L \notin \text{co-RE}$.

\end{document}
