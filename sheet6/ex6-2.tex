%%%% Originalvorlage von Kai Vogelgesang
\documentclass{article}
\usepackage{iftex}
\ifluatex
	\usepackage{fontspec}
\else
	\usepackage[utf8]{inputenc} %Eingabekodierung nach UTF8-Unicode, Erlaubt direkte Eingabe von Umlauten etc.
\fi
\usepackage{amssymb}
\usepackage{amsmath}
\usepackage{amsthm}
\usepackage[ngerman]{babel}
\usepackage{csquotes}

\usepackage[headheight=24pt,heightrounded]{geometry}
\usepackage{graphicx}
\usepackage{titlesec}
\usepackage{fancyhdr}
\usepackage{enumerate}

\usepackage{tikz}
\usetikzlibrary{automata}

\usepackage{xcolor}
\definecolor{urlcolor}{RGB}{0,136,204}
\definecolor{linkcolor}{RGB}{204,0,34}
\usepackage[colorlinks,urlcolor=urlcolor,linkcolor=linkcolor]{hyperref}

% TODOs
\usepackage[colorinlistoftodos,prependcaption,textsize=tiny]{todonotes}

% coloneqq
\usepackage{mathtools}

% Blank lines instead of indented first line
\setlength{\parindent}{0pt}
\setlength{\parskip}{\baselineskip}

\newcommand{\sheet}[1]{
\pagestyle{fancy}
\fancyhead[L]{Introduction to Theoretical Computer Science WS2024/25\\Exercise Sheet #1\\November 1st 2024}
\fancyhead[R]{Akansha Jain (7014299)\\Moustafa Said (7024414)\\Francesco Georg Schmitt (2574611)}

\titleformat{\section}
{\normalfont\Large\bfseries} {E#1.\thesection} {0.5em}{}
\titleformat{\subsection}
{\normalfont\large\bfseries} {}{1em}{}

}

% Hier kommen Vorlesungs- und Dokumentspezifische Makros

\newcommand{\NN}{\mathbb{N}}
\newcommand{\ZZ}{\mathbb{Z}}
\newcommand{\QQ}{\mathbb{Q}}
\newcommand{\RR}{\mathbb{R}}
\newcommand{\CC}{\mathbb{C}}
\newcommand{\inv}{^{-1}}
\newcommand{\lpar}{\left(}
\newcommand{\rpar}{\right)}

\newcommand{\limn}{{\lim_{n\to\infty}}}

\sheet{6}{November 29th, 2024}
\begin{document}

\section{\lstinline|FOR|-programs terminate}


We need to prove that FOR-programs always terminate by showing via structural induction that for every FOR-program \( P \), the function \( \Phi_P \) is total.

\subsection*{Base Case: Simple Assignments}
For a basic assignment, the program always terminates immediately after executing the statement. Hence, \( \Phi_P \) is defined, making it total.

\subsection*{Inductive Hypothesis (IH)}
For any FOR-programs \( P_1 \) and \( P_2 \), the functions \( \Phi_{P_1} \) and \( \Phi_{P_2} \) are total. 
\subsection*{Inductive Step}
If \( P \) is of the form \( [P_1; P_2] \):
\begin{itemize}
    \item By the inductive hypothesis, \( \Phi_{P_1} \) is total, so \( P_1 \) terminates.
    \item After \( P_1 \) terminates, \( P_2 \) starts execution. Again, by the inductive hypothesis, \( P_2 \) terminates, and \( \Phi_{P_2} \) is total.
\end{itemize}
Therefore, the combined program \( P_1; P_2 \) terminates, and \( \Phi_P \) is total.

If \( P \) is a FOR-loop:
\[
P = \text{for } x_i \text{ do } P_1 \text{ od}
\]
the number of iterations is fixed and determined before entering the loop. For the following cases:
\begin{itemize}
    \item If \( x_i = 0 \), the loop body \( P_1 \) does not execute, and the program terminates immediately.
    \item If \( x_i > 0 \), \( P_1 \) executes exactly \( x_i \) times. By the inductive hypothesis, \( \Phi_{P_1} \) is total, so \( P_1 \) terminates after each iteration.
\end{itemize}
Since the number of iterations is finite and \( P_1 \) terminates for each iteration, the loop terminates after \( x \) iterations, then \( \Phi_P \) is total.



\end{document}