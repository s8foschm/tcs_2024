%%%% Originalvorlage von Kai Vogelgesang
\documentclass{article}
\usepackage{iftex}
\ifluatex
	\usepackage{fontspec}
\else
	\usepackage[utf8]{inputenc} %Eingabekodierung nach UTF8-Unicode, Erlaubt direkte Eingabe von Umlauten etc.
\fi
\usepackage{amssymb}
\usepackage{amsmath}
\usepackage{amsthm}
\usepackage[ngerman]{babel}
\usepackage{csquotes}

\usepackage[headheight=24pt,heightrounded]{geometry}
\usepackage{graphicx}
\usepackage{titlesec}
\usepackage{fancyhdr}
\usepackage{enumerate}

\usepackage{tikz}
\usetikzlibrary{automata}

\usepackage{xcolor}
\definecolor{urlcolor}{RGB}{0,136,204}
\definecolor{linkcolor}{RGB}{204,0,34}
\usepackage[colorlinks,urlcolor=urlcolor,linkcolor=linkcolor]{hyperref}

% TODOs
\usepackage[colorinlistoftodos,prependcaption,textsize=tiny]{todonotes}

% coloneqq
\usepackage{mathtools}

% Blank lines instead of indented first line
\setlength{\parindent}{0pt}
\setlength{\parskip}{\baselineskip}

\newcommand{\sheet}[1]{
\pagestyle{fancy}
\fancyhead[L]{Introduction to Theoretical Computer Science WS2024/25\\Exercise Sheet #1\\November 1st 2024}
\fancyhead[R]{Akansha Jain (7014299)\\Moustafa Said (7024414)\\Francesco Georg Schmitt (2574611)}

\titleformat{\section}
{\normalfont\Large\bfseries} {E#1.\thesection} {0.5em}{}
\titleformat{\subsection}
{\normalfont\large\bfseries} {}{1em}{}

}

\input{macros}
\sheet{1}
\begin{document}

\section{Finite Automata}

\subsection{(a)}

$A = \left\{ x \in \left\{0,1\right\} ^{*} \mid \lvert x \rvert \text{ is even} \right\}$

\begin{tikzpicture}[->,node distance=2cm, auto]
\node[initial,state,accepting](A) {even};
\node[state](B) [right of=A] {odd};

\path (A) edge [bend left] node {0,1} (B)
      (B) edge [bend left] node {0,1} (A);
\end{tikzpicture}

We want an automaton to determine whether the length of a word of $\left\{0,1\right\}^{*}$ is an even or odd number. For this, we start with the empty word $\epsilon$, which has length $0$, which is even. For every single digit ($0$ or $1$) we get when our length is currently even, our length becomes odd (because every even number $+1$ becomes odd), and for every digit we add to a word of odd-numbered length, the length becomes even.

\subsection{(b)}

$B = \left\{ x \in \left\{0,1\right\} ^{*} \mid x \text{ contains exactly three } 0s \right\}$

\begin{tikzpicture}[->,node distance=2cm, auto]
\node[initial,state](A) {no 0s};
\node[state](B) [right of=A] {one 0};
\node[state](C) [right of=B] {two 0s};
\node[state,accepting](D) [right of=C] {three 0s};
\node[state](E) [right of=D] {more};

\path (A) edge  node {0} (B)
          edge [loop above] node {1} (A)
      (B) edge  node {0} (C)
          edge [loop above] node {1} (B)
      (C) edge  node {0} (D)
          edge [loop above] node {1} (C)
      (D) edge node {0} (E)
          edge [loop above] node {1} (D)
      (E) edge [loop above] node {0,1} (E);
\end{tikzpicture}

Whenever we read a $1$, we stay in the same state, on the other hand, if we read a $0$, we jump to the next state. After reading exactly 3 $0$s we are in the accepting state, but after reading one more $0$ we go to the state where we have more than 3 $0$s and therefore reject it. Once we have read more than 3 $0$s, we remain in that state no matter the number read.

\subsection{(c)}

$C = \left\{ x \in \left\{0,1\right\} ^{*} \mid x \text{ does not contain two (or more) consecutive } 0s \right\}$

\begin{tikzpicture}[->,node distance=2cm, auto]
\node[initial,state,accepting](A) {no 0s};
\node[state,accepting](B) [right of=A] {one 0};
\node[state](C) [right of=B] {two 0s};


\path (A) edge [bend left]  node {0} (B)
          edge [loop above] node {1} (A)
      (B) edge  node {0} (C)
          edge [bend left] node {1} (A)
      (C) edge [loop above]  node {0,1} (C);
\end{tikzpicture}

Here, we accept $x$ which does not contain two or more $0$s. When we read a $0$, we are still in an accepting state, from which, if followed by another $0$, we go to the rejecting state. If we read a $1$ in the first step we stay in the starting state and if we read a $1$ in the second state, we jump right back to the starting state, since we only want to count consecutive $0$s in this automaton.

\subsection{(d)}

$D = \left\{ x \in \left\{0,1\right\} ^{*} \mid x \text{ mod } 3 = 0 \right\}$

We will construct the automaton by constructing each of the individual edges.\\
For this, we will need some modular arithmetic tools:\\
$n\text{ mod } 3 = 0 \Leftrightarrow \exists k \in \mathbb{N}: n = 3\cdot k$\\
$n\text{ mod } 3 = 1 \Leftrightarrow \exists k \in \mathbb{N}: n = 3\cdot k + 1$\\
$n\text{ mod } 3 = 2 \Leftrightarrow \exists k \in \mathbb{N}: n = 3\cdot k + 2$\\
Also, appending a $0$ to a binary number gives the following: $(n0)_{2} = (2*n)_{10}$\\
Similarly, appending a $1$ to a binary number gives the following: $(n1)_{2} = (2*n+1)_{10}$\\

From this, we get the following transitions:
\begin{itemize}
    \item initial state $n \text{ mod } 3 = 0 \Leftrightarrow \exists k \in \mathbb{N}: n = 3\cdot k$:
    \begin{itemize}
        \item append $0$: $(n0)_{2} = (2n)_{10} = (2\cdot(3k))_{10} = (6k)_{10} = (3(2k))_{10} = (3k')_{10}$
        where $k'=2k \in \mathbb{N}$\\
        $\rightarrow n \text{ mod } 3 = 0 \Leftrightarrow (n0)_{2} \text{ mod } 3 = 0$
        \item append $1$\todo{should I put the index 10 for every single step of the calculation?}: $(n1)_{2} = (2n+1)_{10} = (2\cdot(3k)+1) = 6k+1 = 3(2k')+1$
        where $k'=2k \in \mathbb{N}$\\
        $\rightarrow n \text{ mod } 3 = 0 \Leftrightarrow (n1)_{2} \text{ mod } 3 = 1$
    \end{itemize}
    \item initial state $n \text{ mod } 3 = 1 \Leftrightarrow \exists k \in \mathbb{N}: n = 3\cdot k + 1$:
    \begin{itemize}
        \item append $0$: $(n0)_{2} = (2n)_{10} = (2\cdot(3k+1)) = 6k+2 = 3(2k')+2$
        where $k'=2k \in \mathbb{N}$\\
        $\rightarrow n \text{ mod } 3 = 1 \Leftrightarrow (n0)_{2} \text{ mod } 3 = 2$
        \item append $1$: $(n1)_{2} = (2n+1)_{10} = (2\cdot(3k+1)+1) = 6k+2+1 = 3(k')$
        where $k'=2k+1 \in \mathbb{N}$\\
        $\rightarrow n \text{ mod } 3 = 1 \Leftrightarrow (n1)_{2} \text{ mod } 3 = 0$
    \end{itemize}
    \item initial state $n \text{ mod } 3 = 2 \Leftrightarrow \exists k \in \mathbb{N}: n = 3\cdot k + 2$:
    \begin{itemize}
        \item append $0$: $(n0)_{2} = (2n)_{10} = (2\cdot(3k+2)) = 6k+4 = 3(2k)+3+1 = 3(2k+1)+1 = 3k'+1$
        where $k' = 2k+1\in \mathbb{N}$\\
        $\rightarrow n \text{ mod } 3 = 2 \Leftrightarrow (n0)_{2} \text{ mod } 3 = 1$
        \item append $1$: $(n1)_{2} = (2n+1)_{10} = (2\cdot(3k+2)+1) = 6k+5 = 3(2k)+3+2 = 3(2k+1)+2 = 3k'+2$
        where $k'=2k+1 \in \mathbb{N}$\\
        $\rightarrow n \text{ mod } 3 = 2 \Leftrightarrow (n1)_{2} \text{ mod } 3 = 2$
    \end{itemize}
\end{itemize}

We have now handled all possible transitions and can therefore assemble the automaton:

\begin{tikzpicture}[->,node distance=2cm, auto]
\node[initial,state,accepting](A) {0};
\node[state](B) [right of=A] {1};
\node[state](C) [right of=B] {2};


\path (A) edge [bend left]  node {1} (B)
          edge [loop above] node {0} (A)
      (B) edge [bend left]  node {0} (C)
          edge [bend left]  node {1} (A)
      (C) edge [bend left]  node {0} (B)
          edge [loop above] node {1} (C);
\end{tikzpicture}

\subsection{(e)}

$E = \left\{ x \in \left\{0,1,2,3\right\} ^{*} \mid x = x_1x_2\ldots x_{l-1}x_l \text{ and } \prod_{i=1}^{l} x_i = 12\right\}$,

\begin{tikzpicture}[->,node distance=2cm, auto]
\node[initial,state](A) {1};
\node[state](B) [right of=A] {2};
\node[state](C) [right of=B] {3};
\node[state](D) [right of=C] {4};
\node[state](E) [right of=D] {6};
\node[state,accepting](F) [right of=E] {12};
\node[state](G) [below=of D] {not valid};

\path (A) edge [loop above]  node {1} (A)
          edge  node {2} (B)
          edge [bend left] node {3} (C)
          edge [bend left=-30] node {0} (G)

      (B) edge [loop right]  node {1} (B)
          edge [bend left=80] node {2} (D)
          edge [bend left=110] node {3} (E)
          edge [bend left=-30] node {0} (G)

      (C) edge [loop right]  node {1} (C)
          edge [bend left=80] node {2} (E)
          edge [bend left=-30] node {0,3} (G)
          
      
      (D) edge [loop right]  node {1} (D)
          edge [bend left=90] node {3} (F)
          edge [bend left=-30] node {0,2} (G)
          
           
      (E) edge [loop below]  node {1} (E)
          edge node {2} (F)
          edge [bend left=-30] node {0,3} (G)
          
      (F) edge [loop right]  node {1} (F)
          edge [bend left=30] node {0,2,3} (G)
          
      (G) edge [loop below]  node {0,1,2,3} (G);
\end{tikzpicture}

This finite automata has 1 as starting state and 12 as the accepting state.

STATE 1: stays at the same state for input 1\\
         jumps to state 2 for input 2 \\
         jumps to state 3 for input 3\\
         jumps to not valid for input 0

STATE 2: stays at the same state for input 1\\
         jumps to state 4 for input 2 \\
         jumps to state 6 for input 3 \\
         jumps to state not valid for input 0

STATE 3: stays at the same state for input 1\\
         jumps to state 6 for input 2 \\
         jumps to state not valid for input 3,0 

STATE 4: stays at the same state for input 1 \\
         jumps to state 12(accepting state) for input 3\\ 
         jumps to state not valid for input 0,2

STATE 6: stays at the same state for input 1\\
         jumps to state 12(accepting state) for input 2\\ 
         jumps to state not valid for input 0,3

STATE 12: stays at the same state (accepting state) for input 
          1\\                   
          jumps to state not valid for input 0,2,3

STATE not valid: stays at the same state for input all inputs\\
        
         


\subsection{(f)}

$F = \left\{ x \in \left\{1,2,3\right\} ^{*} \mid \lvert x \rvert \geq 1 \text{ and } x = x_1x_2\ldots x_{l-1}x_l \text{ and } x_1 \geq x_2 \geq \ldots \geq x_{l-1} \geq x_l \right\}$

\begin{tikzpicture}[->,node distance=2cm, auto]
\node[initial,state](A) {S};
\node[state,accepting](B) [right of= A] {2};
\node[state,accepting](C) [above=of B] {1};
\node[state,accepting](D) [below=of B] {3};
\node[state](E) [right of=B] {trap};

\path  (A) edge node {2} (B)
           edge [bend left] node {1} (C)
           edge [bend right] node {3} (D)

       (B) edge [loop below] node {2} (B)
           edge node {1} (C)
           edge [bend left=90] node {3} (E)

       (C) edge [loop above] node {1} (C)
           edge [bend left] node {2,3} (E)

       (D) edge [loop below] node {3} (D)
           edge [bend right] node {2} (B)
           edge [bend right=30] node {1} (C);
\end{tikzpicture}

Here we look for sequences with numbers 1,2 and 3 following certain conditions. The accepting states here are 1,2 and 3 on the other hand, the trap state represents invalid sequences(the sequence that is not according to the rules of accepting state).If we read a 1 from the start state we end up at state 1 (accepting state) which upon the reading of a 2 or a 3 jumps tot the trap state, simlarly for state 2 , reading 2 stays in the same state whereas reading a 1 jumps to state 1 and reading a 3 jumps to trap state.For state 3 , reading a 3 remains in the same state and reading a 2 jumps to state 2 and reading a 1 to state 1.

\end{document}
