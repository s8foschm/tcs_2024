%%%% Originalvorlage von Kai Vogelgesang
\documentclass{article}
\usepackage{iftex}
\ifluatex
	\usepackage{fontspec}
\else
	\usepackage[utf8]{inputenc} %Eingabekodierung nach UTF8-Unicode, Erlaubt direkte Eingabe von Umlauten etc.
\fi
\usepackage{amssymb}
\usepackage{amsmath}
\usepackage{amsthm}
\usepackage[ngerman]{babel}
\usepackage{csquotes}

\usepackage[headheight=24pt,heightrounded]{geometry}
\usepackage{graphicx}
\usepackage{titlesec}
\usepackage{fancyhdr}
\usepackage{enumerate}

\usepackage{tikz}
\usetikzlibrary{automata}

\usepackage{xcolor}
\definecolor{urlcolor}{RGB}{0,136,204}
\definecolor{linkcolor}{RGB}{204,0,34}
\usepackage[colorlinks,urlcolor=urlcolor,linkcolor=linkcolor]{hyperref}

% TODOs
\usepackage[colorinlistoftodos,prependcaption,textsize=tiny]{todonotes}

% coloneqq
\usepackage{mathtools}

% Blank lines instead of indented first line
\setlength{\parindent}{0pt}
\setlength{\parskip}{\baselineskip}

\newcommand{\sheet}[1]{
\pagestyle{fancy}
\fancyhead[L]{Introduction to Theoretical Computer Science WS2024/25\\Exercise Sheet #1\\November 1st 2024}
\fancyhead[R]{Akansha Jain (7014299)\\Moustafa Said (7024414)\\Francesco Georg Schmitt (2574611)}

\titleformat{\section}
{\normalfont\Large\bfseries} {E#1.\thesection} {0.5em}{}
\titleformat{\subsection}
{\normalfont\large\bfseries} {}{1em}{}

}

% Hier kommen Vorlesungs- und Dokumentspezifische Makros

\newcommand{\NN}{\mathbb{N}}
\newcommand{\ZZ}{\mathbb{Z}}
\newcommand{\QQ}{\mathbb{Q}}
\newcommand{\RR}{\mathbb{R}}
\newcommand{\CC}{\mathbb{C}}
\newcommand{\inv}{^{-1}}
\newcommand{\lpar}{\left(}
\newcommand{\rpar}{\right)}

\newcommand{\limn}{{\lim_{n\to\infty}}}

\sheet{3}{November 8th 2024}
\begin{document}
\setcounter{section}{2}

\section{Nonregular languages}

\subsection{(a)}

$$
A = \left\{ 1^{3^{n}} \mid n \in \mathbb{N} \right\}
$$

%Let $n \geq 0$ be given ($n \in N $).\\
%Let $u=v=w=1^{3^{n-1}}$, such that $uvw = 1^{3^{n}} \in A$.\\
%We set $v=xyz$, such that there $s,t,r \in \mathbb{N}$, such that $x=1^{s}, y=1^{t}, z=1^{r}, t\geq 1, s+t+r = 3^{m-1}$

%Now we have $uxy^{i}zw = 1^{3^{n-1}}1^{s}1^{t\cdot i}1^{r}1^{3^{n-1}} = 1^{2\cdot 3^{n-1}}1^{s}1^{r}1^{t\cdot i}$

%Now we chose $i \in N$ with $i>1$, and we want to reach a contradiction.\\
%To do this, we know that $2\cdot 3^{m-1} + s + r + t = 3^{m}$, and we want to show that there is an $i\in \mathbb{N}$ for which there is no $k\in \mathbb{N}$ such that $2\cdot 3^{m-1} + s + r + i \cdot t = 3^{k}$, which would imply that we have constructed a word $uxy^{i}zw$ which is not in $A$.

%Since $t\geq 1$, we know that $s+r+2\cdot t > s+r+t = 3^{m-1}$.\\
%If we can show that $s+r+2\cdot t > 3^{m}$, we would have shown that this expression is larger than one power of $3$, and smaller than the next successive power of $3$, so it can't be equal to a power of $3$.

%Let's start by picking $i=2$. We consider two cases: either $s+r+2\cdot t$ is a power of $3$, or it is not.\\
%If it is not a power of $3$, we have constructed a word which is not part of our initial language $A$, so by the contraposition of the pumping lemma, we have shown that $A$ is not regular.

%If we assume that $s+r+2\cdot t$ is a power of $3$, we chose another value for $i$, namely $i=3$.\\
%By our assumption, we know that $s+r+2\cdot t = s+r+t+t = 3^{n-1} + t \in \left\{3^{k} \mid k \in \mathbb{N}\right\}$\\

Let $n \geq 0$ be given ($n \in N $).\\
Select $u = \epsilon$, $v=1^{3^{n}}$, $v=\epsilon$ such that $uvw = \epsilon 1^{3^{n}} \epsilon = 1^{3^{n}} \in A$\\
We know that $\lvert v \rvert \geq n$.

Let $x,y,z$ be given s.t. $xyz=v$ and $|y| > 0$ and $x=1^{3^{s}},y=1^{3^{t}},z=1^{3^{r}} $ s.t. $ 3^{s}+3^{t}+3^{r} = 3^{n} $ and $ t\geq 1$.

Now we have $uxy^izw = \epsilon 1^{3^{s}} 1^{3^{t\cdot i}} 1^{3^{r}} \epsilon = 1^{3^{s+t+r}} 1^{3^{t(i-1)}} = 1^{3^{n}}1^{3^{t(i-1)}}$

Chosing $i=1 \in \mathbb{N}$, we get $uxy^izw = 1^{3^{n}}1^{3^{t(1-1)}} = 1^{3^{n}}1^{3^{0}} = 1^{3^{n}}1^{1} = 1^{3^{n}}1 \notin A$\\
We know this since $\nexists k \in \mathbb{N}: \forall n \in \mathbb{N}: 3^{n} + 1 = 3^{k}$\todo{is this correct, and should we even mention this?}\\
We have provoked a contradiction. Therefore, A does not fulfill the pumping property and is therefore not regular.$\qed$

\subsection{(b)}

$$
B = \left\{ x \in \left\{0,1\right\}^{\ast} \mid \text{there are more }0\text{ appearing in }x\text{ than }1\right\}
$$

Let $n \geq 0$ be given ($n \in N $).\\
We choose $u = \epsilon$, $v= 1^n$, $w= 0^{n+1}$ s.t. $uvw \in B$ and $|v| = n$.\\
Let $x,y,z$ be given s.t. $xyz=v$ and $|y| > 0$ and $x=1^s,y=1^t,z=1^r $ s.t. $ s+t+r = n $ and $ t \geq 1$.

Now we have $uxy^izw = \epsilon 1^s 1^{t*i} 1^r 0^{n+1} = \epsilon 1^s 1^t 1^r 1^{t * (i-1)} 0^{n+1} = \epsilon 1^n 1^{t(i-1)}0^{n+1}$

Now we chose $i \in N$ with $i > 1$, since we also have $t \geq 1, $ then the term $t(i-1) \geq 1$ for $t \geq 1$ and $i > 1$ and thus the term $1^{t * (i-1)}$ is at least $1^1$. this means for i=2, we have  $uxy^2zw$ = $\epsilon 1^s 1^t 1^r 1^{t * (2-1)} 0^{n+1}$ = $\epsilon 1^n 1^{t}0^{n+1}$ = $\epsilon 1^{n + t}0^{n+1} \notin B$ with $t\geq1$ and thus B does not fulfill the pumping property and is therefore not regular.$\qed$

\end{document}
