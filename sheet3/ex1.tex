%%%% Originalvorlage von Kai Vogelgesang
\documentclass{article}
\usepackage{iftex}
\ifluatex
	\usepackage{fontspec}
\else
	\usepackage[utf8]{inputenc} %Eingabekodierung nach UTF8-Unicode, Erlaubt direkte Eingabe von Umlauten etc.
\fi
\usepackage{amssymb}
\usepackage{amsmath}
\usepackage{amsthm}
\usepackage[ngerman]{babel}
\usepackage{csquotes}

\usepackage[headheight=24pt,heightrounded]{geometry}
\usepackage{graphicx}
\usepackage{titlesec}
\usepackage{fancyhdr}
\usepackage{enumerate}

\usepackage{tikz}
\usetikzlibrary{automata}

\usepackage{xcolor}
\definecolor{urlcolor}{RGB}{0,136,204}
\definecolor{linkcolor}{RGB}{204,0,34}
\usepackage[colorlinks,urlcolor=urlcolor,linkcolor=linkcolor]{hyperref}

% TODOs
\usepackage[colorinlistoftodos,prependcaption,textsize=tiny]{todonotes}

% coloneqq
\usepackage{mathtools}

% Blank lines instead of indented first line
\setlength{\parindent}{0pt}
\setlength{\parskip}{\baselineskip}

\newcommand{\sheet}[1]{
\pagestyle{fancy}
\fancyhead[L]{Introduction to Theoretical Computer Science WS2024/25\\Exercise Sheet #1\\November 1st 2024}
\fancyhead[R]{Akansha Jain (7014299)\\Moustafa Said (7024414)\\Francesco Georg Schmitt (2574611)}

\titleformat{\section}
{\normalfont\Large\bfseries} {E#1.\thesection} {0.5em}{}
\titleformat{\subsection}
{\normalfont\large\bfseries} {}{1em}{}

}

\input{macros}
\sheet{3}{November 8th 2024}
\begin{document}

\section{Introduction to transducers}


\[
M = (Q, \Sigma, \Gamma, \delta, f, q_0, Q_{\text{acc}})
\]
\[
f : Q \times \Sigma \to \Gamma^{\ast}
\]
\subsection{(a)}

\[
\Sigma = \{0, 1\}^{\ast} \quad \text{and} \quad f_{M_1}(x_1 x_2 \cdots x_{2n-1}) = x_1 x_1 x_3 x_3 x_5 x_5 \cdots x_{2n-1} x_{2n-1}
\]
\[
\text{for all } n \in \mathbb{N} \setminus \{0\} \text{ over } \{0, 1\}^{\ast}.
\]

\begin{tikzpicture}[->,node distance=2cm, auto]
\node[initial,state](A) {even};
\node[state,accepting](B) [right of=A] {odd};

\path (A) edge [bend left] node {0:00,1:11} (B)

         (B) edge [bend left] node {$0:\epsilon$,$1:\epsilon$} (A);
\end{tikzpicture}


Only start state is accepting because  \( f_{M_1} \) is undefined if the length of the input is even.



\subsection{(b)}


\[
\Sigma = \{a, b, c, d\}
\]
 \[
 \quad f_{M_2}(x) \text{ outputs the substring of } x \text{ delimited by the first occurrence of } c \text{ and the first subsequent occurrence of } d.
\]

\begin{tikzpicture}[->,node distance=2cm, auto]
\node[initial,state](A) {start};
\node[state](B) [right of=A] {X};
\node[state,accepting](C) [right of=B] {Y};

\path (A) edge node {c:c} (B)
               edge [loop above] node {a,b,$d:\epsilon$} (A)

         (B) edge node {d:d} (C)
              edge [loop above] node {a:a,b:b,c:c} (B)

         (C) edge [loop below] node {a,b,c,$d:\epsilon$} (C);

  
\end{tikzpicture}

Only the last state being accepting is enough to not leave expressions like $abc$ undefined.


\end{document}
