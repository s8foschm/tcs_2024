%%%% Originalvorlage von Kai Vogelgesang
\documentclass{article}
\usepackage{iftex}
\ifluatex
	\usepackage{fontspec}
\else
	\usepackage[utf8]{inputenc} %Eingabekodierung nach UTF8-Unicode, Erlaubt direkte Eingabe von Umlauten etc.
\fi
\usepackage{amssymb}
\usepackage{amsmath}
\usepackage{amsthm}
\usepackage[ngerman]{babel}
\usepackage{csquotes}

\usepackage[headheight=24pt,heightrounded]{geometry}
\usepackage{graphicx}
\usepackage{titlesec}
\usepackage{fancyhdr}
\usepackage{enumerate}

\usepackage{tikz}
\usetikzlibrary{automata}

\usepackage{xcolor}
\definecolor{urlcolor}{RGB}{0,136,204}
\definecolor{linkcolor}{RGB}{204,0,34}
\usepackage[colorlinks,urlcolor=urlcolor,linkcolor=linkcolor]{hyperref}

% TODOs
\usepackage[colorinlistoftodos,prependcaption,textsize=tiny]{todonotes}

% coloneqq
\usepackage{mathtools}

% Blank lines instead of indented first line
\setlength{\parindent}{0pt}
\setlength{\parskip}{\baselineskip}

\newcommand{\sheet}[1]{
\pagestyle{fancy}
\fancyhead[L]{Introduction to Theoretical Computer Science WS2024/25\\Exercise Sheet #1\\November 1st 2024}
\fancyhead[R]{Akansha Jain (7014299)\\Moustafa Said (7024414)\\Francesco Georg Schmitt (2574611)}

\titleformat{\section}
{\normalfont\Large\bfseries} {E#1.\thesection} {0.5em}{}
\titleformat{\subsection}
{\normalfont\large\bfseries} {}{1em}{}

}

% Hier kommen Vorlesungs- und Dokumentspezifische Makros

\newcommand{\NN}{\mathbb{N}}
\newcommand{\ZZ}{\mathbb{Z}}
\newcommand{\QQ}{\mathbb{Q}}
\newcommand{\RR}{\mathbb{R}}
\newcommand{\CC}{\mathbb{C}}
\newcommand{\inv}{^{-1}}
\newcommand{\lpar}{\left(}
\newcommand{\rpar}{\right)}

\newcommand{\limn}{{\lim_{n\to\infty}}}

\sheet{5}{November 22nd, 2024}
\begin{document}
\setcounter{section}{2}

\section{Even more syntactical sugar}

To achieve this, we need the following:
\begin{enumerate}
    \item We need to keep track that \texttt{return x} is called at most once.
    \item The returned value \texttt{x} must be saved in \texttt{$x_0$} and never overwritten afterward.
\end{enumerate}

To achieve this, we introduce two variables that may only be changed by a \texttt{return x} statement: \texttt${x_{returned}$ := 0;} and \texttt{$x_{return\_val}$}, and this must be done before entering the loop.

Now, when simulating a \texttt{return x} in a \textbf{WHILE} program \texttt{P = while $x_i \neq 0$ do $P_1$ od}, the \texttt{return x} statement itself is simulated as follows:

\begin{verbatim}
if x_{returned} = 0 then
    x_0 := 0;
    x_{return\_val} := x + x_0;
    x_{returned} := 1;
    x_i := 0;
else
    x_{dummy\_variable} := 0;
fi;
\end{verbatim}

In this statement, \texttt{$x_i$} is set to 0 to ensure the \texttt{WHILE} loop terminates after this iteration. We want our return value to be stored in \texttt{$x_0$}, so we must ensure that the final modification to \texttt{$x_0$} is the value of \texttt{x}. To implement this in both \texttt{FOR} and \texttt{WHILE} programs, the last part of the loop must be:

\begin{verbatim}
if x_{returned} != 0 then
    x_0 := 0;
    x_0 := x_{return\_val} + x_0;
else
    x_{dummy\_variable}++;
fi;
\end{verbatim}

In contrast to a \texttt{WHILE} program, a \textbf{FOR} program \texttt{P = for $x_i$ do $P_1$ od} will not terminate before completing $x_i$ iterations. Therefore, the statement \texttt{$x_i$ := 0} is not needed in a \texttt{FOR} program. The simulation of the \texttt{return x} statement itself would then be:

\begin{verbatim}
if x_{returned} = 0 then
    x_0 := 0;
    x_{return\_val} := x + x_0;
    x_{returned} := 1;
else
    x_{dummy\_variable} := 0;
fi;
\end{verbatim}

\end{document}
