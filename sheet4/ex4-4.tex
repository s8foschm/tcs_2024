%%%% Originalvorlage von Kai Vogelgesang
\documentclass{article}
\usepackage{iftex}
\ifluatex
	\usepackage{fontspec}
\else
	\usepackage[utf8]{inputenc} %Eingabekodierung nach UTF8-Unicode, Erlaubt direkte Eingabe von Umlauten etc.
\fi
\usepackage{amssymb}
\usepackage{amsmath}
\usepackage{amsthm}
\usepackage[ngerman]{babel}
\usepackage{csquotes}

\usepackage[headheight=24pt,heightrounded]{geometry}
\usepackage{graphicx}
\usepackage{titlesec}
\usepackage{fancyhdr}
\usepackage{enumerate}

\usepackage{tikz}
\usetikzlibrary{automata}

\usepackage{xcolor}
\definecolor{urlcolor}{RGB}{0,136,204}
\definecolor{linkcolor}{RGB}{204,0,34}
\usepackage[colorlinks,urlcolor=urlcolor,linkcolor=linkcolor]{hyperref}

% TODOs
\usepackage[colorinlistoftodos,prependcaption,textsize=tiny]{todonotes}

% coloneqq
\usepackage{mathtools}

% Blank lines instead of indented first line
\setlength{\parindent}{0pt}
\setlength{\parskip}{\baselineskip}

\newcommand{\sheet}[1]{
\pagestyle{fancy}
\fancyhead[L]{Introduction to Theoretical Computer Science WS2024/25\\Exercise Sheet #1\\November 1st 2024}
\fancyhead[R]{Akansha Jain (7014299)\\Moustafa Said (7024414)\\Francesco Georg Schmitt (2574611)}

\titleformat{\section}
{\normalfont\Large\bfseries} {E#1.\thesection} {0.5em}{}
\titleformat{\subsection}
{\normalfont\large\bfseries} {}{1em}{}

}

\input{macros}
\sheet{4}{November 15, 2024}

\begin{document}

\setcounter{section}{3} % Set the section counter to 3 so the next section is 4
\renewcommand{\thesection}{\arabic{section}} % Section numbering without prefix

\section{Infinite Unions}

\subsection{(a)}
Let \( B_{\cup^n} = \bigcup_{k \in \{0, \dots, n\}} \{0^k\} \). We can use the Myhill-Nerode Theorem to show that all equivalence classes \( [0^{k-i}]_{B_{\cup^n}} \) for \( k \in \{0, \dots, n\} \), \( 0 \leq i \leq k \), and any fixed \( n \in \mathbb{N} \) with \( 0 \leq k \leq n \) are pairwise distinct. From this, we conclude that the minimal DFA that recognizes \( B_{\cup^n} \) must have at least \( n + 1 \) states.

For any \( i \neq j \) where \( 0 \leq i \leq k \) and \( 0 \leq j \leq k \), we have \( 0^{k-i} \not\sim_{B_{\cup^n}} 0^{k-j} \) because \( 0^{k-i} 0^i \in B_{\cup^n} \) but \( 0^{k-j} 0^i \notin B_{\cup^n} \).

Similarly, let \( A_{\cup^n} = \bigcup_{k \in \{0, \dots, n\}} \{0^k 1^k\} \). We can also show that all equivalence classes \( [0^k 1^{k-i}]_{A_{\cup^n}} \) for \( k \in \{0, \dots, n\} \), \( 0 \leq i \leq k \), and any fixed \( n \in \mathbb{N} \) with \( 0 \leq k \leq n \) are pairwise distinct. This implies that the minimal DFA recognizing \( A_{\cup^n} \) must have at least \( n + 1 \) states.

For any \( i \neq j \), we have \( 0^k 1^{k-i} \not\sim_{A_{\cup^n}} 0^k 1^{k-j} \) because \( 0^k 1^{k-i} 1^i \in A_{\cup^n} \) but \( 0^k 1^{k-j} 1^i \notin A_{\cup^n} \).

\subsection{(b)}
The language \( A := \bigcup_{k \in \mathbb{N}} A_k \) is not regular. This is because, for any \( i \neq j \) where \( 0 \leq i \leq k \) and \( 0 \leq j \leq k \), we have \( 0^k 1^{k-i} \not\sim_{A} 0^k 1^{k-j} \) since \( 0^k 1^{k-i} 1^i \in A \) but \( 0^k 1^{k-j} 1^i \notin A \) for all \( k \in \mathbb{N} \). Since \( k \) is arbitrary, the equivalence classes \( [0^k 1^{k-i}]_{\sim_{A}} \) are pairwise distinct for arbitrary values of \( k \). Thus, the index of \( \sim_{A} \) is infinite, which implies that \( A \) is not regular.

On the other hand, \( B := \bigcup_{k \in \mathbb{N}} B_k = \bigcup_{k \in \mathbb{N}} \{0^k\} \) is regular since we can construct the following simple DFA to recognize \( B \):

\begin{center}
\begin{tikzpicture}[shorten >=1pt, node distance=2cm, on grid, auto]
   \node[state, initial, accepting] (q0) {$B$};
   \path (q0) edge[loop above] node{0} (q0);
\end{tikzpicture}
\end{center}

\subsection{(c)}

\begin{enumerate}
    \item[\textbf{(i)}] That is not true, a counter example is what we have seen in the last 2 Subproblems where $A_k := \{0^k1^k\}$ is finite and therefore regular but \( A := \bigcup_{k \in \mathbb{N}} A_k \) is not regular as shown in exericse (b).

    \item[\textbf{(ii)}] We want to prove or disprove that $\forall i \in \NN: L_{i} \in \text{REG} \Rightarrow \bigcup_{i\in \NN}  L_{i} \in \text{REG}$\\
    Let L be a non-regular language: $L \notin \text{REG}$\\
    We know that $\left\{w\right\}$ is a regular language (since it is finite) $\forall w \in \Sigma^{\ast}\setminus \left\{w\right\}$.\\
    We know that regular languages are closed over complement (since they are closed over the set difference and the complement of any regular language $L$ is defined as $\Sigma^{\ast}\setminus L$, and $\Sigma^{\ast}$ is regular), so this means that $\Sigma^{\ast}\setminus \left\{w\right\}$ is also regular.\\
    We construct a set as follows:\\
    $\bigcap_{w\notin L}(\Sigma^{\ast}\setminus\left\{w\right\}) = L$
    So L is a non-regular languages which could be constructed as an intersection of ininitely many regular languages.\\
    Therefore, we have shown that the initial statement does not hold.$\square$
\end{enumerate}


\end{document}
