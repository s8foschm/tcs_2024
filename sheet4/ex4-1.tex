%%%% Originalvorlage von Kai Vogelgesang
\documentclass{article}
\usepackage{iftex}
\ifluatex
	\usepackage{fontspec}
\else
	\usepackage[utf8]{inputenc} %Eingabekodierung nach UTF8-Unicode, Erlaubt direkte Eingabe von Umlauten etc.
\fi
\usepackage{amssymb}
\usepackage{amsmath}
\usepackage{amsthm}
\usepackage[ngerman]{babel}
\usepackage{csquotes}

\usepackage[headheight=24pt,heightrounded]{geometry}
\usepackage{graphicx}
\usepackage{titlesec}
\usepackage{fancyhdr}
\usepackage{enumerate}

\usepackage{tikz}
\usetikzlibrary{automata}

\usepackage{xcolor}
\definecolor{urlcolor}{RGB}{0,136,204}
\definecolor{linkcolor}{RGB}{204,0,34}
\usepackage[colorlinks,urlcolor=urlcolor,linkcolor=linkcolor]{hyperref}

% TODOs
\usepackage[colorinlistoftodos,prependcaption,textsize=tiny]{todonotes}

% coloneqq
\usepackage{mathtools}

% Blank lines instead of indented first line
\setlength{\parindent}{0pt}
\setlength{\parskip}{\baselineskip}

\newcommand{\sheet}[1]{
\pagestyle{fancy}
\fancyhead[L]{Introduction to Theoretical Computer Science WS2024/25\\Exercise Sheet #1\\November 1st 2024}
\fancyhead[R]{Akansha Jain (7014299)\\Moustafa Said (7024414)\\Francesco Georg Schmitt (2574611)}

\titleformat{\section}
{\normalfont\Large\bfseries} {E#1.\thesection} {0.5em}{}
\titleformat{\subsection}
{\normalfont\large\bfseries} {}{1em}{}

}

% Hier kommen Vorlesungs- und Dokumentspezifische Makros

\newcommand{\NN}{\mathbb{N}}
\newcommand{\ZZ}{\mathbb{Z}}
\newcommand{\QQ}{\mathbb{Q}}
\newcommand{\RR}{\mathbb{R}}
\newcommand{\CC}{\mathbb{C}}
\newcommand{\inv}{^{-1}}
\newcommand{\lpar}{\left(}
\newcommand{\rpar}{\right)}

\newcommand{\limn}{{\lim_{n\to\infty}}}

\sheet{4}{November 15th, 2024}
\begin{document}

\section{Minimal DFA}
The language \( L = \{ x \in \{0, 1\}^\star \mid x \text{ ends in one of the substrings } 00 \text{ or } 11 \} \) is recognized by the following automaton:

% Insert the DFA diagram using tikz if necessary.
\begin{center}
\begin{tikzpicture}[shorten >=1pt, node distance=3cm, on grid, auto]
    \node[state, initial] (s)   {$\epsilon$};
    \node[state] (0) [above right=of s] {one 0};
    \node[state] (1) [below right=of s] {one 1};
    \node[state, accepting] (11) [right=of 1] {$11$};
    \node[state, accepting] (00) [right=of 0] {$00$};

    \path[->]
    (s)  edge                 node                  {1} (1)
         edge                 node                  {0} (0)
    (0)  edge                 node                  {0} (00)
         edge [bend left=20]  node [left]           {1} (1)
    (1)  edge                 node                  {1} (11)
         edge [bend left=20]  node [right]          {0} (0)
    (00) edge [loop right]    node                  {0} (00)
         edge                 node [right, pos=0.3] {1} (1)
    (11) edge [loop right]    node                  {1} (11)
         edge                 node [right, pos=0.3] {0} (0);
\end{tikzpicture}
\end{center}

\section{Representatives for Myhill-Nerode Equivalence Classes}

\begin{itemize}
    \item \( [\epsilon] := \emptyset \)
    \item \( [00] := \{ x \mid x \text{ ends with } "00" \} \)
    \item \( [11] := \{ x \mid x \text{ ends with } "11" \} \)
    \item \( [0] := \{ x \mid x \text{ ends with } "10" \text{ and } |x| > 1, \text{ or } x = "0" \} \)
    \item \( [1] := \{ x \mid x \text{ ends with } "01" \text{ and } |x| > 1, \text{ or } x = "1" \} \)
\end{itemize}

\section{Proof of Minimality}
We show that the states are all pairwise Myhill-Nerode inequivalent by constructing a table. For each pair of states, there exists a string \( z \in \Sigma^\star \) that, when appended to each representative, results in one string being in \( L \) and the other not. This proves the minimality of the DFA recognizing \( L \) (up to isomorphism $:)$ ).

\begin{center}
\begin{tabular}{|c|c|c|c|c|c|}
    \hline
    & \([\epsilon]\) & \([0]\) & \([00]\) & \([1]\) & \([11]\) \\
    \hline
    \([\epsilon]\) & - & - & - & - & - \\
    \hline
    \([0]\) & 0 & - & - & - & - \\
    \hline
    \([00]\) & 0 & \( \epsilon \) & - & - & - \\
    \hline
    \([1]\) & 1 & 1 & 1 & - & - \\
    \hline
    \([11]\) & 1 & 1 & 1 & \( \epsilon \) & - \\
    \hline
\end{tabular}
\end{center}

\end{document}
