%%%% Originalvorlage von Kai Vogelgesang
\documentclass{article}
\usepackage{iftex}
\ifluatex
	\usepackage{fontspec}
\else
	\usepackage[utf8]{inputenc} %Eingabekodierung nach UTF8-Unicode, Erlaubt direkte Eingabe von Umlauten etc.
\fi
\usepackage{amssymb}
\usepackage{amsmath}
\usepackage{amsthm}
\usepackage[ngerman]{babel}
\usepackage{csquotes}

\usepackage[headheight=24pt,heightrounded]{geometry}
\usepackage{graphicx}
\usepackage{titlesec}
\usepackage{fancyhdr}
\usepackage{enumerate}

\usepackage{tikz}
\usetikzlibrary{automata}

\usepackage{xcolor}
\definecolor{urlcolor}{RGB}{0,136,204}
\definecolor{linkcolor}{RGB}{204,0,34}
\usepackage[colorlinks,urlcolor=urlcolor,linkcolor=linkcolor]{hyperref}

% TODOs
\usepackage[colorinlistoftodos,prependcaption,textsize=tiny]{todonotes}

% coloneqq
\usepackage{mathtools}

% Blank lines instead of indented first line
\setlength{\parindent}{0pt}
\setlength{\parskip}{\baselineskip}

\newcommand{\sheet}[1]{
\pagestyle{fancy}
\fancyhead[L]{Introduction to Theoretical Computer Science WS2024/25\\Exercise Sheet #1\\November 1st 2024}
\fancyhead[R]{Akansha Jain (7014299)\\Moustafa Said (7024414)\\Francesco Georg Schmitt (2574611)}

\titleformat{\section}
{\normalfont\Large\bfseries} {E#1.\thesection} {0.5em}{}
\titleformat{\subsection}
{\normalfont\large\bfseries} {}{1em}{}

}

% Hier kommen Vorlesungs- und Dokumentspezifische Makros

\newcommand{\NN}{\mathbb{N}}
\newcommand{\ZZ}{\mathbb{Z}}
\newcommand{\QQ}{\mathbb{Q}}
\newcommand{\RR}{\mathbb{R}}
\newcommand{\CC}{\mathbb{C}}
\newcommand{\inv}{^{-1}}
\newcommand{\lpar}{\left(}
\newcommand{\rpar}{\right)}

\newcommand{\limn}{{\lim_{n\to\infty}}}

\sheet{4}{November 15th 2024}

\begin{document}
\setcounter{section}{2}

\section{Pumping Lemma vs. Myhill-Nerode}

$$L = \left\{ x \in \left\{0,1\right\}^{\ast} \mid 11\text{ is a subword of }x\text{ or } \exists n \in \mathbb{N}:\#_{1}(x)=n^{2}\right\}$$

\subsection{(a)}
Let \( n = 4 \).

For all \( uvw \in L \): \( |v| \geq n \).

\textbf{Case 1: \( uvw \in L \) because it contains \( 11 \)}

(i) If \( v \) contains only 1s:

Choose \( y = 11 \). Since \( n = 4 \) and we are in the case where \( v \) contains only 1s, there are still two untouched consecutive 1s. This way, if you pump down or pump up, you still have two consecutive 1s. Formally, if \( v = xyz = 1111(1^*) \) such that \( |v| \geq n \), then by choosing \( y := 11 \), we have:
\[
\forall i \in \mathbb{N}, \quad uxy^i zw = ux(11)^i zw \in L
\]

(ii) If \( v \) contains \( 11 \) and at least one "0" (\( \#_0(v) > 0 \)):

Choose \( y = 0 \), then we have:
\[
\forall i \in \mathbb{N}, \quad uxy^i zw = ux(0)^i zw \in L
\]

(iii) If \( v \) doesn’t contain \( 11 \):

Since \( |v| \geq n \), this implies there must be a zero. Choose \( y = 0 \). Again, since \( |v| \geq n \), we have:
\[
\forall i \in \mathbb{N}, \quad uxy^i zw = ux(0)^i zw \in L
\]

\textbf{Case 2: \( uvw \in L \) because \( \exists n \in \mathbb{N} \) such that \( \#_{1}(uvw) = n^2 \) and there is no \( 11 \)}

We have \( n = 4 \), \( |v| \geq n \), and \( uvw \) doesn’t contain \( 11 \). This implies \( v \) contains either only 0s or alternating 1s and 0s. Choose \( y = 0 \). Since \( |v| \geq n \), we have:
\[
\forall i \in \mathbb{N}, \quad uxy^i zw = ux(0)^i zw \in L
\]

Therefore, we have shown that \( L \) satisfies the pumping lemma in all cases.

\subsection{(b)}
We now want to show that L is regular by directly applying the Myhill-Nerode Theorem to L.\\
We do that by showing that $\sim_{L}$ has infinitely many equivalence classes.\\
The Myhill-Nerode relation is defined as follows:\\
$x\sim_{L}y= [\forall z \in \sigma^{\ast}: xz \in L \Leftrightarrow yz \in L]$\\
Let $i,j \in \mathbb{N}, i \neq j$. Pick 2 equivalence classes and show they're distinct.\\
We pick:
\vspace{-5mm}
\begin{itemize}
    \item $0^{i}1$
    \item $0^{j}10^{j}$
\end{itemize}
\vspace{-5mm}
We append $(10)^{k}$ to both of these expressions, with $\forall n \in \mathbb{N}: k+1 \neq n^{2}$.\\
We then get:
\vspace{-5mm}
\begin{itemize}
    \item $0^{i}1(10)^k \in L$ since it contains the string 11
    \item $0^{j}10^{j}(10)^k \notin L$ since it doesn't contain 11 at any point, and it contains $k+1$ 1s, where $k+1 \neq n^{2}$ due to our definition of $k$
\end{itemize}
\vspace{-5mm}
Therefore $[0^{i}]_{\sim_{L}}$ and $[0^{j}10^{j}]_{\sim_{L}}$ are pairwise distinct equivalence classes of $\sim_{L}$ $\forall i,j \in \mathbb{N}$.\\
This proves that $\sim_{L}$ has an infinite number of equivalence classes, which makes $L$ not regular according to the Myhill-Nerode Theorem. $\square$


\end{document}
