%%%% Originalvorlage von Kai Vogelgesang
\documentclass{article}
\usepackage{iftex}
\ifluatex
	\usepackage{fontspec}
\else
	\usepackage[utf8]{inputenc} %Eingabekodierung nach UTF8-Unicode, Erlaubt direkte Eingabe von Umlauten etc.
\fi
\usepackage{amssymb}
\usepackage{amsmath}
\usepackage{amsthm}
\usepackage[ngerman]{babel}
\usepackage{csquotes}

\usepackage[headheight=24pt,heightrounded]{geometry}
\usepackage{graphicx}
\usepackage{titlesec}
\usepackage{fancyhdr}
\usepackage{enumerate}

\usepackage{tikz}
\usetikzlibrary{automata}

\usepackage{xcolor}
\definecolor{urlcolor}{RGB}{0,136,204}
\definecolor{linkcolor}{RGB}{204,0,34}
\usepackage[colorlinks,urlcolor=urlcolor,linkcolor=linkcolor]{hyperref}

% TODOs
\usepackage[colorinlistoftodos,prependcaption,textsize=tiny]{todonotes}

% coloneqq
\usepackage{mathtools}

% Blank lines instead of indented first line
\setlength{\parindent}{0pt}
\setlength{\parskip}{\baselineskip}

\newcommand{\sheet}[1]{
\pagestyle{fancy}
\fancyhead[L]{Introduction to Theoretical Computer Science WS2024/25\\Exercise Sheet #1\\November 1st 2024}
\fancyhead[R]{Akansha Jain (7014299)\\Moustafa Said (7024414)\\Francesco Georg Schmitt (2574611)}

\titleformat{\section}
{\normalfont\Large\bfseries} {E#1.\thesection} {0.5em}{}
\titleformat{\subsection}
{\normalfont\large\bfseries} {}{1em}{}

}

% Hier kommen Vorlesungs- und Dokumentspezifische Makros

\newcommand{\NN}{\mathbb{N}}
\newcommand{\ZZ}{\mathbb{Z}}
\newcommand{\QQ}{\mathbb{Q}}
\newcommand{\RR}{\mathbb{R}}
\newcommand{\CC}{\mathbb{C}}
\newcommand{\inv}{^{-1}}
\newcommand{\lpar}{\left(}
\newcommand{\rpar}{\right)}

\newcommand{\limn}{{\lim_{n\to\infty}}}

\sheet{2}{November 1st 2024}
\begin{document}

\setcounter{section}{2}
\section{Homomorphisms}

\subsection{(a)}
We are indeed convinced that that $h(\epsilon) = \epsilon$ and that $h$ is uniquely determined given $h(\sigma)$ for every $\sigma \in \Sigma$.

\subsection{(b)}
$A=L(a*b*c)$\\
$B=L((01)*2)$\\

Construct the homomorphism $h$ such that\todo{do we need to prove that this is indeed a homomorphism and/or that it transforms A into B??}:
\begin{align*}
    h(\epsilon)&=\epsilon\\
    h(a)&=01\\
    h(b)&=01\\
    h(c)&=2\\
\end{align*}

\subsection{(c)}

Let $K$ be the language given by the automaton:\\
\begin{tikzpicture}[->,node distance=2cm, auto]
    \node[initial, state]   (S)              {S};
    \node[state]            (A) [right of=S] {A};
    \node[state, accepting] (B) [right of=A] {B};

    \path (S) edge             node {0}   (A)
          (A) edge             node {0,1} (B)
          (B) edge [bend left] node {1}   (S);
\end{tikzpicture}

The language $h(K)$ is going to be given by the automaton:\\
\begin{tikzpicture}[->, node distance=2cm, auto]
    \node[initial, state]   (S)                    {S};
    \node[state]            (M) [above right of=S] {M};
    \node[state]            (A) [below right of=M] {A};
    \node[state]            (N) [above right of=A] {N};
    \node[state, accepting] (B) [below right of=N] {B};

    \path (S) edge             node {a} (M)
          (M) edge             node {b} (A)
          (A) edge             node {a} (N)
          (N) edge             node {b} (B)
          (A) edge             node {a} (B)
          (B) edge [bend left] node {a} (S);

\end{tikzpicture}

The Automaton recognizing $j^-1(k)$: \\

\begin{tikzpicture}[->,node distance=2cm, auto]
    \node[initial, state]   (S)              {S};
    \node[state]            (A) [right of=S] {A};
    \node[state, accepting] (B) [right of=A] {B};

    \path (S) edge             node {a}   (A)
              edge [bend left]             node {b}   (B)
          (A) edge             node {a}   (B)
              edge   [bend left]          node {b}   (S);
\end{tikzpicture}

\subsection{(d)}

Let $C \subseteq \Sigma *$ be a regular language. Show that $h(C)$ is also a regular language.

We use the following construction:\\
Let $M=(Q,\Sigma,\delta,q_{0},Q_{acc})$ be a finite automaton and $L(M)=C$.\\
We construct the automaton $M'=(Q',\Gamma,\delta', q_{0}, Q_{acc})$.

$Q'=Q\cup \{ q_{s,a,i}\mid s\in Q, a \in \Sigma, i = \lvert h(a) \rvert \} $\\
For every $h(a)$ with $a \in \Sigma$, we add $\lvert h(a)\rvert -1$ additional states.\\
Our alphabet is $\Gamma$, as given in the exercise.\\
We define our transition function $\delta'$ as follows:\\
\begin{align*}
\delta' (q, w_{k}) &= q_{w_{k},1} &q\in Q\\
\delta' (q_{a,i}, w_{k}) &= 
\begin{cases}
    q_{a,i+1} &i < \lvert a \rvert\\
    \delta(q, w_{k})&i=\lvert a \rvert
\end{cases}
&q\notin Q
\end{align*}
We keep the same starting state $q_{0}$ and accepting states $Q_{acc}$.\\



\subsection{(e)}
Let \( M = (Q, \Sigma, \delta, q_0, Q_{acc}) \) be an automaton that accepts the language \( D \) such that \( L(M) = D \).

\textbf{Intuition:} If we are in state \( q \in Q \) and read a symbol \( a \in \Gamma \) on \( M' \) (the automaton we are constructing for \( h^{-1}(D) \)), we want to reach the state \( p \in Q \) that reading \( h(a) \in \Sigma^* \) on \( M \) from state \( q \) would bring us. This can be formalized as follows:

Construct \( M' = (Q, \Gamma, \delta', q_0, Q_{acc}) \), where \( \delta'(q, x) = \delta^*(q, h(x)) \) such that \( L(M') = h^{-1}(D) \).

\textbf{Proof:}
\begin{itemize}
    \item[$\supseteq$:] Let \( h(a) \in h^{-1}(D) \), then \( \delta^*(q_0, h(a)) \in Q_{\text{acc}} \).
    
    Since \( h \) has the property that \( h(xy) = h(x) h(y) \) for \( x, y \in \text{domain}(h) \) (where domain\( (h) \) is the domain of \( h \)), we can write \( h(a) = h(a_1) h(a_2) \dots h(a_{|a|}) \) where \( a = a_1 a_2 \dots a_{|a|} \). Using this along with Lemma 1.5, we have:
    
    \[
    \delta^*(q_0, h(a)) \in Q_{\text{acc}}
    \]
    \[
    \Leftrightarrow \delta^*(\delta^*(\dots(\delta^*(q_0, h(a_0)), \dots, h(a_{|a|-1})), h(a_{|a|})) \in Q_{\text{acc}}
    \]
    
    By the construction of \( M' \),
    
    \[
    \Rightarrow \delta'(\delta'(\dots(\delta'(q_0, a_0), \dots, a_{|a|-1}), a_{|a|}) \in Q_{\text{acc}}
    \]
    
    Using Lemma 1.5,
    
    \[
    \Leftrightarrow \delta'^*(q_0, a) \in Q_{\text{acc}}
    \]
    \[
    \Rightarrow a \in L(M')
    \]

    \item[$\subseteq$:] Now, let \( a \in L(M') \), where \( a = a_1 a_2 \dots a_{|a|} \).
    
    \[
    \Rightarrow \delta'^*(q_0, a) \in Q_{\text{acc}}
    \]

    By definition of \( \delta^* \),

    \[
    \Leftrightarrow \delta'(\delta'(\dots(\delta'(q_0, a_0), \dots, a_{|a|-1}), a_{|a|}) \in Q_{\text{acc}}
    \]

    By the construction of \( M' \),

    \[
    \Rightarrow \delta^*(\delta^*(\dots(\delta^*(q_0, h(a_0)), \dots, h(a_{|a|-1})), h(a_{|a|})) \in Q_{\text{acc}}
    \]

    By Lemma 1.5,

    \[
    \Rightarrow \delta^*(q_0, h(a)) \in Q_{\text{acc}}
    \]
    \[
    \Rightarrow h(a) \in h^{-1}(D)
    \]

    \qedsymbol
\end{itemize}



\end{document}
