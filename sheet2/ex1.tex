%%%% Originalvorlage von Kai Vogelgesang
\documentclass{article}
\usepackage{iftex}
\ifluatex
	\usepackage{fontspec}
\else
	\usepackage[utf8]{inputenc} %Eingabekodierung nach UTF8-Unicode, Erlaubt direkte Eingabe von Umlauten etc.
\fi
\usepackage{amssymb}
\usepackage{amsmath}
\usepackage{amsthm}
\usepackage[ngerman]{babel}
\usepackage{csquotes}

\usepackage[headheight=24pt,heightrounded]{geometry}
\usepackage{graphicx}
\usepackage{titlesec}
\usepackage{fancyhdr}
\usepackage{enumerate}

\usepackage{tikz}
\usetikzlibrary{automata}

\usepackage{xcolor}
\definecolor{urlcolor}{RGB}{0,136,204}
\definecolor{linkcolor}{RGB}{204,0,34}
\usepackage[colorlinks,urlcolor=urlcolor,linkcolor=linkcolor]{hyperref}

% TODOs
\usepackage[colorinlistoftodos,prependcaption,textsize=tiny]{todonotes}

% coloneqq
\usepackage{mathtools}

% Blank lines instead of indented first line
\setlength{\parindent}{0pt}
\setlength{\parskip}{\baselineskip}

\newcommand{\sheet}[1]{
\pagestyle{fancy}
\fancyhead[L]{Introduction to Theoretical Computer Science WS2024/25\\Exercise Sheet #1\\November 1st 2024}
\fancyhead[R]{Akansha Jain (7014299)\\Moustafa Said (7024414)\\Francesco Georg Schmitt (2574611)}

\titleformat{\section}
{\normalfont\Large\bfseries} {E#1.\thesection} {0.5em}{}
\titleformat{\subsection}
{\normalfont\large\bfseries} {}{1em}{}

}

% Hier kommen Vorlesungs- und Dokumentspezifische Makros

\newcommand{\NN}{\mathbb{N}}
\newcommand{\ZZ}{\mathbb{Z}}
\newcommand{\QQ}{\mathbb{Q}}
\newcommand{\RR}{\mathbb{R}}
\newcommand{\CC}{\mathbb{C}}
\newcommand{\inv}{^{-1}}
\newcommand{\lpar}{\left(}
\newcommand{\rpar}{\right)}

\newcommand{\limn}{{\lim_{n\to\infty}}}

\sheet{2}{November 1st 2024}
\begin{document}

\section{Regular expressions}

\subsection{(a)}

$A = \left\{ x \in \left\{0,1\right\} ^{*} \mid \lvert x \rvert \text{ is even} \right\}$

$A = L(((0+1)(0+1))^*)$

Kleene Closure of either empty or contains the an arbitrary number of 2 symbols.

\subsection{(b)}

$B = \left\{ x \in \left\{0,1\right\} ^{*} \mid x \text{ contains \textit{exactly} three } 0s \right\}$

$B = L(1^*01^*01^*01^*)$
We accept 3x 0's and any amount of 1's between and around them.

\subsection{(c)}

$C = \left\{ x \in \left\{0,1\right\} ^{*} \mid x \text{ does not contain two (or more) consecutive } 0s \right\}$

$C = L(((01)^*+(10)^*+1^*) + 0)$
We accept the kleene closure of either a sequence of 01's or 10's followed by a 1 or an arbitrary amount of 1's 
\subsection{(d)}

$D = \left\{ x \in \left\{0,1\right\} ^{*} \mid x \text{ \texttt{mod} } 3 = 0 \right\}$

$D = L(0*(1(1+01^*01))^*)$

The regular expression is derived by the general algorithm shown in the lecture on the automaton of E1.1 d) :

\begin{tikzpicture}[->,node distance=2cm, auto]
\node[initial,state,accepting](A) {0};
\node[state](B) [right of=A] {1};
\node[state](C) [right of=B] {2};


\path (A) edge [bend left]  node {1} (B)
          edge [loop above] node {0} (A)
      (B) edge [bend left]  node {0} (C)
          edge [bend left]  node {1} (A)
      (C) edge [bend left]  node {0} (B)
          edge [loop above] node {1} (C);
\end{tikzpicture}

\subsection{(e)}

$E = \left\{ x \in \left\{0,1,2,3\right\} ^{*} \mid x = x_1x_2\ldots x_{l-1}x_l \text{ and } \prod_{i=1}^{l} x_i = 12\right\}$

$E = L(1^*((21^*21^*31^*)+(21^*31^* + 31^*21^*)2)1*)$

By observing the Automaton in E1.1 e):

\begin{tikzpicture}[->,node distance=2cm, auto]
\node[initial,state](A) {1};
\node[state](B) [right of=A] {2};
\node[state](C) [right of=B] {3};
\node[state](D) [right of=C] {4};
\node[state](E) [right of=D] {6};
\node[state,accepting](F) [right of=E] {12};
\node[state](G) [below=of D] {not valid};

\path (A) edge [loop above]  node {1} (A)
          edge  node {2} (B)
          edge [bend left] node {3} (C)
          edge [bend left=-30] node {0} (G)

      (B) edge [loop right]  node {1} (B)
          edge [bend left=80] node {2} (D)
          edge [bend left=110] node {3} (E)
          edge [bend left=-30] node {0} (G)

      (C) edge [loop right]  node {1} (C)
          edge [bend left=80] node {2} (E)
          edge [bend left=-30] node {0,3} (G)
          
      
      (D) edge [loop right]  node {1} (D)
          edge [bend left=90] node {3} (F)
          edge [bend left=-30] node {0,2} (G)
          
           
      (E) edge [loop below]  node {1} (E)
          edge node {2} (F)
          edge [bend left=-30] node {0,3} (G)
          
      (F) edge [loop right]  node {1} (F)
          edge [bend left=30] node {0,2,3} (G)
          
      (G) edge [loop below]  node {0,1,2,3} (G);
\end{tikzpicture}

We can build this Expression by following the possible paths of the Automaton backwards from the accepting state 12. 
\subsection{(f)}

$F = \left\{ x \in \left\{1,2,3\right\} ^{*} \mid \lvert x \rvert \geq 1 \text{ and } x = x_1x_2\ldots x_{l-1}x_l \text{ and } x_1 \geq x_2 \geq \ldots \geq x_{l-1} \geq x_l \right\}$

$F = L(3^*2^*1^*)$

The expression preserves descending order of possible values {1,2,3}
\end{document}
