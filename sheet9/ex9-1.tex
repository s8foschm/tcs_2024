%%%% Originalvorlage von Kai Vogelgesang
\documentclass{article}
\usepackage{iftex}
\ifluatex
	\usepackage{fontspec}
\else
	\usepackage[utf8]{inputenc} %Eingabekodierung nach UTF8-Unicode, Erlaubt direkte Eingabe von Umlauten etc.
\fi
\usepackage{amssymb}
\usepackage{amsmath}
\usepackage{amsthm}
\usepackage[ngerman]{babel}
\usepackage{csquotes}

\usepackage[headheight=24pt,heightrounded]{geometry}
\usepackage{graphicx}
\usepackage{titlesec}
\usepackage{fancyhdr}
\usepackage{enumerate}

\usepackage{tikz}
\usetikzlibrary{automata}

\usepackage{xcolor}
\definecolor{urlcolor}{RGB}{0,136,204}
\definecolor{linkcolor}{RGB}{204,0,34}
\usepackage[colorlinks,urlcolor=urlcolor,linkcolor=linkcolor]{hyperref}

% TODOs
\usepackage[colorinlistoftodos,prependcaption,textsize=tiny]{todonotes}

% coloneqq
\usepackage{mathtools}

% Blank lines instead of indented first line
\setlength{\parindent}{0pt}
\setlength{\parskip}{\baselineskip}

\newcommand{\sheet}[1]{
\pagestyle{fancy}
\fancyhead[L]{Introduction to Theoretical Computer Science WS2024/25\\Exercise Sheet #1\\November 1st 2024}
\fancyhead[R]{Akansha Jain (7014299)\\Moustafa Said (7024414)\\Francesco Georg Schmitt (2574611)}

\titleformat{\section}
{\normalfont\Large\bfseries} {E#1.\thesection} {0.5em}{}
\titleformat{\subsection}
{\normalfont\large\bfseries} {}{1em}{}

}

% Hier kommen Vorlesungs- und Dokumentspezifische Makros

\newcommand{\NN}{\mathbb{N}}
\newcommand{\ZZ}{\mathbb{Z}}
\newcommand{\QQ}{\mathbb{Q}}
\newcommand{\RR}{\mathbb{R}}
\newcommand{\CC}{\mathbb{C}}
\newcommand{\inv}{^{-1}}
\newcommand{\lpar}{\left(}
\newcommand{\rpar}{\right)}

\newcommand{\limn}{{\lim_{n\to\infty}}}

\sheet{9}{December 20th, 2024}
\begin{document}

\section{Index Sets}

\subsection*{(a)}
$L_1$ is an index set.  

Let $i \in L_1, j \in \mathbb{N}$ with $\varphi_i = \varphi_j$.  

\[
i \in L_1 \implies \varphi_i(1337) = 0  \text{ (def } L_1).
\]

\[
\varphi_i = \varphi_j \implies \varphi_j(1337) = 0.
\]

\[
\implies j \in L_1, \text{ and } L_1 \text{ is an index set.}\text{ (def } L_1)  \quad \Box
\]

\subsection*{(b)}
$L_2$ is not an index set.

\[
\exists g: \text{im } \varphi_g = \{0, \ldots, g\} \text{ (}\implies \text{im } \varphi_g \subseteq \{0, \ldots, g\} \text{ holds), } g \in L_2.
\]

\[
\exists k \in \mathbb{N}: k > g \text{ and } \varphi_g = \varphi_k.
\]

\[
\varphi_g = \varphi_k \implies \text{im } \varphi_k = \{0, \ldots, g\}.
\]

But since $k > g \implies \text{im } \varphi_k \neq \{0, \ldots, k\}$:

\[
\implies k \notin L_2
\]

\[
\implies L_2 \text{ not index set.} \quad \Box
\]

\subsection*{(c)}
$L_3$ is not an index set.

Let $i \in L_3, j \in \mathbb{N}$ with $\varphi_i = \varphi_j$ and $i \neq j$.  

\[
i \in L_3 \implies i = 42.
\]

We have $\varphi_i = \varphi_j$, but $i \neq j \implies j \neq 42 \implies j \notin L_3$.  

Thus, $L_3$ is not an index set. \quad $\Box$

\subsection*{(d)}
For any $i \in \mathbb{N}$, $M_i := \{j \in \mathbb{N} \mid \varphi_i = \varphi_j\}$ is and index set.  

Let $i \in \mathbb{N}, j \in M_i$, and $k \in \mathbb{N}$ with $\varphi_j = \varphi_k$.  

\[
j \in M_i \implies \varphi_i = \varphi_j = \varphi_k \implies k \in M_i.
\]

This proves $M_i$ is an index set for any $i \in \mathbb{N}$. \quad $\Box$

\end{document}
