%%%% Originalvorlage von Kai Vogelgesang
\documentclass{article}
\usepackage{iftex}
\ifluatex
	\usepackage{fontspec}
\else
	\usepackage[utf8]{inputenc} %Eingabekodierung nach UTF8-Unicode, Erlaubt direkte Eingabe von Umlauten etc.
\fi
\usepackage{amssymb}
\usepackage{amsmath}
\usepackage{amsthm}
\usepackage[ngerman]{babel}
\usepackage{csquotes}

\usepackage[headheight=24pt,heightrounded]{geometry}
\usepackage{graphicx}
\usepackage{titlesec}
\usepackage{fancyhdr}
\usepackage{enumerate}

\usepackage{tikz}
\usetikzlibrary{automata}

\usepackage{xcolor}
\definecolor{urlcolor}{RGB}{0,136,204}
\definecolor{linkcolor}{RGB}{204,0,34}
\usepackage[colorlinks,urlcolor=urlcolor,linkcolor=linkcolor]{hyperref}

% TODOs
\usepackage[colorinlistoftodos,prependcaption,textsize=tiny]{todonotes}

% coloneqq
\usepackage{mathtools}

% Blank lines instead of indented first line
\setlength{\parindent}{0pt}
\setlength{\parskip}{\baselineskip}

\newcommand{\sheet}[1]{
\pagestyle{fancy}
\fancyhead[L]{Introduction to Theoretical Computer Science WS2024/25\\Exercise Sheet #1\\November 1st 2024}
\fancyhead[R]{Akansha Jain (7014299)\\Moustafa Said (7024414)\\Francesco Georg Schmitt (2574611)}

\titleformat{\section}
{\normalfont\Large\bfseries} {E#1.\thesection} {0.5em}{}
\titleformat{\subsection}
{\normalfont\large\bfseries} {}{1em}{}

}

% Hier kommen Vorlesungs- und Dokumentspezifische Makros

\newcommand{\NN}{\mathbb{N}}
\newcommand{\ZZ}{\mathbb{Z}}
\newcommand{\QQ}{\mathbb{Q}}
\newcommand{\RR}{\mathbb{R}}
\newcommand{\CC}{\mathbb{C}}
\newcommand{\inv}{^{-1}}
\newcommand{\lpar}{\left(}
\newcommand{\rpar}{\right)}

\newcommand{\limn}{{\lim_{n\to\infty}}}

\sheet{10}{January 10th, 2025}
\begin{document}

\section{Quiz}

\subsection{(a)}
If $A$ is an index set and $i,j\in A$, then $\varphi_{i}=\varphi_{j}$ holds.\\
The statement is \textbf{False}.\\
\textsc{To-Do.}

\subsection{(b)}
Any language $A\subseteq\NN$ that is not a non-trivial index set is decidable.\\
The statement is \textbf{False}.\\
%Rice's Theorem (Theorem 15.8) does not make any statement about sets which are not index sets.
The Special Halting Problem $H_{0}$ is non-decidable, but it is not an index set according to example 15.8 and the corresponding proof in the script. So, it is a counter-example towards the statement.

\subsection{(c)}
The complement $\overline{A}$ of a finite language $A\subseteq\NN$ is decidable.\\
The statement is \textbf{True}.\\
{\color{red} if we can use this, then it is to be proven in exercise 12.3 on page 114 in the script that $REC$ is closed under complementation.}\\
Every finite set is obviously decidable. By checking whether a number $k$ is not in $A$ (which is \textsc{WHILE}-computable since $A$ is finite), we can find out whether $k$ is in $\overline{A}$.

\subsection{(d)}
For any pair $A,B\in\NN$, either $A\leq B$ or $B\leq A$ holds.\\
The statement is \textbf{False}.\\
Counter-example: we use $A=H_{0}$, $B=\overline{H_{0}}$. We know that neither $H_{0}\leq\overline{H_{0}}$ nor $\overline{H_{0}}\leq H_{0}$ hold.

\subsection{(e)}
Any countable set $A\subseteq\NN$ is recursively enumerable.\\
The statement is \textbf{False}.\\
Since $\NN$ is countable, every subset $A\subseteq\NN$ is also countable.\\
$\overline{H_{0}}$ is a subset of $\NN$ but not recursively enumerable (corollary 12.7).

\subsection{(f)}
There is a set $A\subseteq\NN$ such that $A\leq H_{0}$ and $H_{0}\leq \overline{A}$.\\
The statement is \textbf{False}.\\
If this were true, it would imply $H_{0} \in REC$.{\color{red} expand!}

\subsection{(g)}
The only finite index set is the empty set.\\
The statement is \textbf{True}.\\
There is an infinite number of programs/Turing machines and their respective Gödel numbers which compute a specific function. So, every non-empty index set has to be infinite.

\subsection{(h)}
There is a Turing Machine $M$ that, given the binary representation of a Gödel Number $g$ of a \textsc{WHILE} program, terminates with a $1$ on its tape if $g$ halts with input $g$ and otherwise terminates with $0$ on its tape.\\
The statement is \textbf{False}.\\
This corresponds to the special halting problem $H_{0}$, which is not decidable (Theorem 12.1). Since Turing-computability is equivalent to \textsc{WHILE}-computability (Theorem 15.8), this statement is false.


\end{document}
