%%%% Originalvorlage von Kai Vogelgesang
\documentclass{article}
\usepackage{iftex}
\ifluatex
	\usepackage{fontspec}
\else
	\usepackage[utf8]{inputenc} %Eingabekodierung nach UTF8-Unicode, Erlaubt direkte Eingabe von Umlauten etc.
\fi
\usepackage{amssymb}
\usepackage{amsmath}
\usepackage{amsthm}
\usepackage[ngerman]{babel}
\usepackage{csquotes}

\usepackage[headheight=24pt,heightrounded]{geometry}
\usepackage{graphicx}
\usepackage{titlesec}
\usepackage{fancyhdr}
\usepackage{enumerate}

\usepackage{tikz}
\usetikzlibrary{automata}

\usepackage{xcolor}
\definecolor{urlcolor}{RGB}{0,136,204}
\definecolor{linkcolor}{RGB}{204,0,34}
\usepackage[colorlinks,urlcolor=urlcolor,linkcolor=linkcolor]{hyperref}

% TODOs
\usepackage[colorinlistoftodos,prependcaption,textsize=tiny]{todonotes}

% coloneqq
\usepackage{mathtools}

% Blank lines instead of indented first line
\setlength{\parindent}{0pt}
\setlength{\parskip}{\baselineskip}

\newcommand{\sheet}[1]{
\pagestyle{fancy}
\fancyhead[L]{Introduction to Theoretical Computer Science WS2024/25\\Exercise Sheet #1\\November 1st 2024}
\fancyhead[R]{Akansha Jain (7014299)\\Moustafa Said (7024414)\\Francesco Georg Schmitt (2574611)}

\titleformat{\section}
{\normalfont\Large\bfseries} {E#1.\thesection} {0.5em}{}
\titleformat{\subsection}
{\normalfont\large\bfseries} {}{1em}{}

}

\input{macros}
\sheet{7}{December 6th, 2024}
\setcounter{section}{3}
\usepackage{algorithm, algpseudocode, listings}

\begin{document}

\section{\lstinline|RE| and \lstinline|co-RE|}

\subsection{(a)}

$L_1 \in \text{RE}$ because we can compute $\chi'_{L_1}$ with the following \textbf{WHILE} program. Assume that $P_i := \text{göd}^{-1}(i)$:

\begin{algorithm}
    \caption{input: i}
    \begin{algorithmic}[1]
        \State $X_0 := 1337$
        \State $P_1$
        \While{$X_0 \neq 42$}
            \State $X_0 := 0$
        \EndWhile
        \State $X_0 := 1$
    \end{algorithmic}
\end{algorithm}

\noindent
Since this \textbf{WHILE} program computes $\chi'_{L_1}$, we have that indeed, $L_1 \in \text{RE}$.

\subsection{(b)}
$L_1 \notin \text{co-RE}$

\noindent
We know that $\overline{H_0} \notin \text{RE} \implies H_0 \notin \text{co-RE}$.  
We can show that $H_0 \leq L_1$ with the following program:

\[
L_1 = \{i \mid \textbf{göd}^{-1}(i) \text{outputs 42 on input 1337} \}
\]
\[
H_0 = \{g \mid g_{\text{öd}}(g) \text{ halts on input } g\}.
\]

We want that $\chi'_{H_0} = \chi'_{L_1} o f$   
\[
f: \mathbb{N} \to \mathbb{N}, \quad f(g) := \textbf{göd}(P),
\]
where $P$ is the following \textbf{WHILE} program and $P_2 = \textbf{göd}^{-1}(g)$:

\begin{algorithm}
    \caption{input: j}
    \begin{algorithmic}[1]
        \State $X_0 := g$
        \State $P_2$
        \State $X_0 := 42$
    \end{algorithmic}
\end{algorithm}

\noindent
If $g \notin H_0$, then $P_2 = \textbf{göd}^{-1}(g)$ halts on input $g$,  
and $P$ outputs $42$ on input $j$, which would be $1337$,  
and thus $f(g) = \textbf{göd}(P) \in L_1$.

\noindent
If $g \notin H_0$, then $P_2 = \textbf{göd}^{-1}(g)$ doesn’t halt on input $g$, and $P$ diverges.  
Thus, $f(g) = \textbf{göd}(P) \notin L_1$.

\noindent
Thus, the mapping $f$ that maps $g \mapsto \textbf{göd}^{-1}(P)$ is the desired reduction.

\bigskip

\noindent
We now show that $f$ is \textbf{WHILE-computable}:

\[
g \mapsto \langle 4, \langle \langle 2, \langle 0, g \rangle \rangle_7 \rangle_5, \langle 4, \langle g, \langle 2, \langle 0, 42 \rangle_5 \rangle_7 \rangle_5 \rangle_5 \rangle.
\]

\noindent
The above mapping is \textbf{WHILE-computable}, because the pairing functions are \textbf{WHILE-computable}.

this shows that $L_1 \notin \textbf{co-RE} $ 

\subsection{(c)}

We prove that $L_2 \notin \text{RE}$ because there exists a reduction $\overline{H_0} \leq L_2$.  
Since $\overline{H_0} \notin \text{RE}$, it follows that $L_2 \notin \text{RE}$.

\bigskip

We want a function $f: \mathbb{N} \to \mathbb{N}$ such that $\chi'_{\overline{H_0}} = \chi'_{L_2} \circ f$.  
We have:
\[
L_2 = \{i \mid \text{göd}^{-1}(i) \text{ diverges on every input}\},
\]
\[
\overline{H_0} = \{g \mid \text{göd}^{-1}(g) \text{ doesn’t halt on input } g\}.
\]

Let $P_1 = \text{göd}^{-1}(g)$ and consider the following \textbf{WHILE}-program $P$:

\begin{algorithm}
    \caption{Input j}
    \begin{algorithmic}[1]
        \State $X_0 = g$
        \State  $P_1$
        \State $X_0 = 1$
    \end{algorithmic}
\end{algorithm}

The reduction $f$ is given by:
\[
f(g) = \text{göd}(P) = \text{göd}([\![X_0 := g; P_1;] X_0 := 1;]\!]).
\]
$f$ is obviously \textbf{WHILE}-computable.
\noindent
Note that P completely ignores its input j and depends only on g.
\bigskip

\noindent

\begin{itemize}
    \item If $g \in \overline{H_0}$, then $P_1$ doesn’t halt on input $g$, and $P$ diverges. Thus, $f(g) = \text{göd}(P) \in L_2$.
    \item If $g \notin \overline{H_0}$, then $P_1$ halts on input $g$, and $P$ doesn’t diverge. Thus, $f(g) = \text{göd}(P) \notin L_2$.
\end{itemize}

This shows that indeed $\overline{H_0} \leq L_2$, and therefore, since $\overline{H_0} \notin \text{RE}$, it holds that $L_2 \notin \text{RE}$.

\bigskip
\bigskip

\subsection{(d)}
$L_2 \in \text{co-RE}$ because $\overline{L_2} \in \text{RE}$.  
We show this by proving that $\chi'_{\overline{L_2}}$ is \textbf{WHILE}-computable with the following \textbf{WHILE}-program $P$, which takes as input $i$:



\noindent
Simulate $i$ on all possible inputs for a fixed number of steps $S$.  
If any of the simulations terminate, output $1$.  
If no more inputs are available, simulate $i$ on all possible inputs for $2 * S$ steps.

\bigskip

\noindent
$P$ is obviously \textbf{WHILE}-computable and computes $\chi'_{\overline{L_2}}$,  
therefore $L_2 \in \text{co-RE}$.

\end{document}
