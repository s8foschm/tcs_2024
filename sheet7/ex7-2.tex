%%%% Originalvorlage von Kai Vogelgesang
\documentclass{article}
\usepackage{iftex}
\ifluatex
	\usepackage{fontspec}
\else
	\usepackage[utf8]{inputenc} %Eingabekodierung nach UTF8-Unicode, Erlaubt direkte Eingabe von Umlauten etc.
\fi
\usepackage{amssymb}
\usepackage{amsmath}
\usepackage{amsthm}
\usepackage[ngerman]{babel}
\usepackage{csquotes}

\usepackage[headheight=24pt,heightrounded]{geometry}
\usepackage{graphicx}
\usepackage{titlesec}
\usepackage{fancyhdr}
\usepackage{enumerate}

\usepackage{tikz}
\usetikzlibrary{automata}

\usepackage{xcolor}
\definecolor{urlcolor}{RGB}{0,136,204}
\definecolor{linkcolor}{RGB}{204,0,34}
\usepackage[colorlinks,urlcolor=urlcolor,linkcolor=linkcolor]{hyperref}

% TODOs
\usepackage[colorinlistoftodos,prependcaption,textsize=tiny]{todonotes}

% coloneqq
\usepackage{mathtools}

% Blank lines instead of indented first line
\setlength{\parindent}{0pt}
\setlength{\parskip}{\baselineskip}

\newcommand{\sheet}[1]{
\pagestyle{fancy}
\fancyhead[L]{Introduction to Theoretical Computer Science WS2024/25\\Exercise Sheet #1\\November 1st 2024}
\fancyhead[R]{Akansha Jain (7014299)\\Moustafa Said (7024414)\\Francesco Georg Schmitt (2574611)}

\titleformat{\section}
{\normalfont\Large\bfseries} {E#1.\thesection} {0.5em}{}
\titleformat{\subsection}
{\normalfont\large\bfseries} {}{1em}{}

}

% Hier kommen Vorlesungs- und Dokumentspezifische Makros

\newcommand{\NN}{\mathbb{N}}
\newcommand{\ZZ}{\mathbb{Z}}
\newcommand{\QQ}{\mathbb{Q}}
\newcommand{\RR}{\mathbb{R}}
\newcommand{\CC}{\mathbb{C}}
\newcommand{\inv}{^{-1}}
\newcommand{\lpar}{\left(}
\newcommand{\rpar}{\right)}

\newcommand{\limn}{{\lim_{n\to\infty}}}

\usepackage{graphicx}
\usepackage{xcolor}
\usepackage{listings}
\usepackage{tikz}

\lstdefinelanguage{pseudo}{
    morekeywords={for, while, if, else, fi},
    sensitive=true,
    morecomment=[l]{//},
    morestring=[b]"
}

% Custom style for code
\lstdefinestyle{pseudo}{
    language=pseudo,
    basicstyle=\ttfamily\small, % Font style
    keywordstyle=\color{blue}, % Color for keywords
    commentstyle=\color{gray}, % Color for comments
    stringstyle=\color{red}, % Color for strings
    numbers=left, % Line numbers
    numberstyle=\tiny\color{black},
    stepnumber=1,
    numbersep=10pt,
    showstringspaces=false,
    breaklines=true,
    frame=single,
    morekeywords=[2]{Input, Output},
    keywordstyle=[2]\color{orange},
    morekeywords=[3]{do, od},
    keywordstyle=[3]\color{purple},
}

\sheet{7}{December 6th, 2024}
\setcounter{section}{1}
\begin{document}

\section{Closure properties}

\subsection{(a)}
REC is closed under complementation, intersection, and union.

\textbf{Proof:}

Let $A, B \in \text{REC}$, then by definition of REC, $\chi_A$ and $\chi_B$ are \textbf{WHILE}-computable and total functions, which means they will always terminate.

We define and implement $\chi_{\overline{A}}, \chi_{A \cap B}, \chi_{A \cup B}$ as follows:

\[
\chi_{\overline{A}}(x) :=
\begin{cases}
    0 & \text{if } \chi_A(x) = 1, \\
    1 & \text{if } \chi_A(x) = 0.
\end{cases}
\]

$\chi_{\overline{A}}$ is obviously \textbf{WHILE}-computable because $\chi_A$ is \textbf{WHILE}-computable.

\bigskip

Since $A, B \in \text{REC}$, there exist \textbf{WHILE}-programs that compute $\chi_A$ and $\chi_B$.

Let $P_A$ be a \textbf{WHILE}-program that computes $\chi_A$, and $P_B$ a \textbf{WHILE}-program that computes $\chi_B$. The following \textbf{WHILE}-program $P$ computes $\chi_{A \cap B}$:

\bigskip

\textbf{Input:} $x$
\[
x_0 := x;
\]
\[
P_A;
\]
\[
\text{if } x_0 = 0 \text{ then}
\]
\[
\quad \text{Return } 0;
\]
\[
\text{else}
\]
\[
\quad P_B;
\]
\[
\text{fi;}
\]

Therefore, $\chi_{A \cap B}$ is \textbf{WHILE}-computable, and $A \cap B \in \text{REC}$.

\bigskip

For union, $\chi_{A \cup B}$ can be implemented as:
\[
\chi_{A \cup B}(x) :=
\begin{cases}
    1 & \text{if } \chi_A(x) = 1 \text{ or } \chi_B(x) = 1, \\
    0 & \text{otherwise}.
\end{cases}
\]

The following \textbf{WHILE}-program computes $\chi_{A \cup B}$:

\bigskip

\textbf{Input:} $x$
\[
x_0 := x;
\]
\[
P_A;
\]
\[
P_B;
\]
\bigskip

\noindent
Therefore, $\chi_{A \cup B}$ is \textbf{WHILE}-computable, and $A \cup B \in \text{REC}$.


\subsection{(b)}


RE is closed under union and intersection but not under complementation.

\textbf{Proof:}

A counterexample of complementation is $\overline{H_0} \notin \text{RE}$ but $H_0 \in \text{RE}$.

We define:
\[
\chi'_{A \cap B}(x) :=
\begin{cases}
    1 & \text{if } \chi'_A(x) = 1 \text{ and } \chi'_B(x) = 1, \\
    \text{undefined} & \text{otherwise}.
\end{cases}
\]

Let $A, B \in \text{RE}$, then $\chi'_A$ and $\chi'_B$ are \textbf{WHILE}-computable.  
We can simulate $\chi'_{A \cap B}$ as follows:

Simulate $\chi'_A$ and $\chi'_B$ in a simultaneous fashion (that is, one program runs one step, then the other program runs one step, and so on) and return $1$ if both $\chi'_A$ and $\chi'_B$ return $1$.

\bigskip

In the same way, we define:
\[
\chi'_{A \cup B}(x) :=
\begin{cases}
    1 & \text{if } \chi'_A(x) = 1 \text{ or } \chi'_B(x) = 1, \\
    \text{undefined} & \text{otherwise}.
\end{cases}
\]

We also implement $\chi'_{A \cup B}$ the same way as we implemented $\chi'_{A \cap B}$, but instead of returning $1$ only when both $\chi'_A$ and $\chi'_B$ simulations return $1$, we instead return $1$ once one of the simulations returns $1$.

\bigskip

This shows that both $\chi'_{A \cap B}$ and $\chi'_{A \cup B}$ are \textbf{WHILE}-computable, and therefore $A \cap B \in \text{RE}$ and $A \cup B \in \text{RE}$.  
Thus, RE is closed under intersection and union.

\end{document}
